

%/******************************************************************************
% *filename:		mokuaisheji.tex
% *author:  		synckey
% *version: 		v1.0
% *datetime:		2011-06-24 08:42:26
% *description:		模块设计
% *****************************************************************************/
\section{模块设计}
\subsection{线程池设计} 
在系统开始运行时创建MAXTHREAD个线程,放入线程池中。主线程创建监听套接字,开启服
务。每个线程各自调用accept,采用互斥锁来保证在每一个时刻只有一个线程调用accept。
每个线程在accept后为客户端服务。
\par{由于TCP内部为监听套接字维护两个队列:a.已完成队列,b.未
完成队列。所以在主线程睡眠期间,新的客户连接会使这两个队列充满(两个队列之和不超
过在listen时指定的backlog),而在这两个队列满了之后当一个客户的SYN到达时,TCP就
忽略该分节,也就时说,并不返回RST。这样做是因为:这种情况是暂时的,客户将重发SYN
,期望在不久就能在这些队列中找到可用空间\cite{unpv1}。 所以当这两个队列充满以后
,客户只是重发SYN ,而服务器不会接受客户端创建更多的连接。}
\begin{figure}[H]
\centering
\includegraphics[keepaspectratio, scale=0.4]{pitures/xianchengmoxing.png}
\caption{多线程模型} 
\end{figure}
\subsection{工作线程设计}

线程池中的工作线程首先调用connect,接受一个客户端的连接,然后从报文的前4个字节中
获取报文的长度,从内存管理模块申请响应大小的内存,用于存放请求报文。
在收到请求后,逐次扫描每个域名。获得域名后,首先在本地的缓存中查找域名对应的IP,
如果有记录,就把IP地址填写到返回报文中,如果没有记录,则填入127.0.0.1,表明本地
没有缓存,将会进行DNS查询,结果在第二个报文中返回。然后把这个域名相应的序号和域
名填入到下图所示的链表中。

\begin{figure}[H]
\centering
\includegraphics[keepaspectratio, scale=0.5]{pitures/response2_dns_list.png}
\caption{本地缓存未命中的域名构成的链表} 
\end{figure}

\par{工作线程在把请求报文遍历完毕后,就把第一次返回报文发送
给客户端,同时也把所有本地没有缓存的域名构造成了一个链表。如果链表为空,说明请求
的域名在本地都有缓存,任务完毕,关闭连接。否则,根据链表中的数目,申请用于构造第
二次返回报文的内存空间。对链表进行遍历,对链表中的每个域名调用一个DNS查询线程进行
DNS查询,在最后一个DNS查询完成后,把第二个报文发送给客户端,关闭连接。}

\begin{figure}[H]
\centering
\includegraphics[keepaspectratio, scale=1]{pitures/xianchengliuchengtu.png}
\caption{线程工作流程图} 
\end{figure}


%%%%%%%%%%%%%%%%%%%%%%%%%%%%%%%%%%%%%%%%%%%%%%%%%
%                                               %
%filename	mm.tex				%
%author		wakemecn			%
%date		6/23/2011			%
%                                               %
%%%%%%%%%%%%%%%%%%%%%%%%%%%%%%%%%%%%%%%%%%%%%%%%%

%\documentclass{article}
%\usepackage{xeCJK}
%\setmainfont[BoldFont=SimHei]{SimSun%}
%\setmonofont{SimSun}
%\XeTeXlinebreaklocale"zh"
%\begin{document}
\subsection{内存管理模块}
		模块从系统预先申请256B,512B,1KB,2KB,$\ldots$,$2^n$KB等2幂次大小的块
		各若干。通过chunk\_manager结构维护特定大小的内存块大小的幂级,预分配的数目,
		已被使用的块链表和空闲的块链表。指针数组维护所有大小块的chunk\_manager结构
		指针。当申请使用M KB($2^m < M < 2^n$)空间时,若$2^n$KB大小块仍有空
		闲块,则将其分配,并将该块移动至已使用块链表中;若上述大小的块不存
		在空闲块或者不存在该大小的块,则模块动态申请若干数目该大小块的内存区域,
		分配内存。当释放某大小块时,若检测到释放的块位于模块动态申请的内存中
		时,则释放该内存。
\begin{figure}[H]
\centering
\includegraphics[keepaspectratio,scale=0.5]{pitures/mm.png}
\caption{内存管理示意图}
\end{figure}


%%%%%%%%%%%%%%%%	end of mm.tex    %%%%%%%%%%%%%%%





%在系统开始运行时初始化clifd数组,其大小为MAXCLI(同时服务的客户端总个数)。令
%iput为要存入clifd的下一个元素的下标,iget为线程池中某个线程将从该数组中取出的下
%一个元素的下标。
%\begin{figure}[H]
%\centering
%\includegraphics[keepaspectratio, scale=0.5]{pitures/xianchengduilie.png}
%\caption{DNS缓存系统子系统层次图}
%\end{figure}
%\par{在主线程中accept客户端的请求,将已连接套接字的文件描述符fd存入clifd,同时检
%查iput有没有赶上iget,如果赶上iget,表明个线程池中的线程处理速度赶不上客户连接的
%速度。则主线程睡眠300us,由于TCP内部为监听套接字维护两个队列:a.已完成队列,b.未
%完成队列。所以在主线程睡眠期间,新的客户连接会使这两个队列充满(两个队列之和不超
%过在listen时指定的backlog),而在这两个队列满了之后当一个客户的SYN到达时,TCP就
%忽略该分节,也就时说,并不返回RST。这样做是因为:这种情况是暂时的,客户将重发SYN
%,期望在不久就能在这些队列中找到可用空间。\cite{unpv1}}
%\par{主线程和线程池中的线程通过条件变量进行同步。}
\subsection{缓存管理模块} 



%/*********************************  END OF mokuaisheji.tex  *********************************/
