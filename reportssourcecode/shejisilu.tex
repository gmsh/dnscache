%
%
%/******************************************************************************
% *filename:		shejisilu.tex
% *author:  		synckey
% *version: 		v1.0
% *datetime:		2011-06-24 14:21:17
% *description:
% *****************************************************************************/
\section{设计思路及折衷} 
\subsection{TCP OR UDP?}
由于可以在一个请求中接受多个域名的查询,所以数据包可能会非常大。UDP缺乏流控制和
错误控制,在数据量比较小的时候使用UDP非常合适。我们一个请求和应答的数据包可
能达到几十K,考虑到客户端的网络流量,如果使用UDP,肯定会发生错误或丢包,在应用程
序中处理错误和重传会造成资源浪费,加重服务器的负担,也会造成网络的拥堵,同时也加
大了客户的响应时间。
\par{而在数据量比较小的时候,使用UDP就非常合适。所以我们初步设计为提供两个服务器
,一个UDP服务器,和一个TCP服务器。客户端根据它的报文大小来选择使用UDP还是TCP。}
\subsection{为什么是LIRS而不是LRU?}
根据资料显示,LIRS的效率比LRU高,但是实现的代价比LRU高不了多少,所以使用LIRS。
\subsection{为什么是字典树,而不使用HASH函数?} 
采用字典树与HASH函数相比,有以下几点优势:
\begin{asparaenum}[1.]
\item{可以利用字符串的公共前缀来节约存储空间} 
\item{查找效率比HASH高\cite{DATA}}
\item{HASH可能产生碰撞} 
\end{asparaenum}
\subsection{为什么使用两个双数组字典树?}
字典树的查询效率非常高,但是动态构造的过程非常慢,需要做大量的内存拷贝。在设计中使用
两个双数组字典树。把访问频度最高的几百万(可配置的数目)个域名存放在文件中,在系
统启动时,从文件中读入域名信息,用于构造主字典树(在已知大小时,字典树构造速度非常
快,因为不需要内存的动态分配和拷贝操作)。在查询域名时,首先从主字典树中查询,如
果未命中,就把相应域名加入副字典树中(副字典树比较小,动态构造小的字典树代价很低)。
这种方式,既保留了字典树的较高的查询速率,又避免了动态构造很大的字典树带来的开销
。

%
%/*********************************  END OF section{设计思路及折衷}  *********************************/
