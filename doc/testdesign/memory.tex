%
%
%/********************************************************************
% * filename:		cache.tex
% * author: 		synckey
% * version:		v1.0
% * datetime:		2011-07-05 13:32:37
% * description:	cache module
% ********************************************************************/
\section{内存管理模块测试用例设计}
\subsection{内存管理初始化功能测试用例}

\begin{table}[H]
\centering
\begin{tabular}{|p{8em}|p{22em}|}
\hline
\textbf{测试项目编号:}&dc\_mm/CASE1\\
\hline
\textbf{测试项目描述:}&测试内存管理模块初始化\\
\hline
\textbf{测试说明:}& 测试dc\_mm初始化功能,预分配特定大小的chunk各若干\\
\hline
\textbf{测试步骤及输入:}&配置随机的chunks大小以及数目,调用mm\_init(),无输入\\
\hline
\textbf{期待结果及输出:}&通过GDB查看内存,相关chunk已经正确初始化\\
\hline
\textbf{测试结果:}& 初始化正常\\
\hline
\textbf{测试所花时间(单位:分钟)}& 约160分钟\\
\hline
\textbf{备注:}& \\
\hline
\end{tabular}
\end{table}

\subsection {内存分配功能测试用例}
\begin{table}[H]
\centering
\begin{tabular}{|p{8em}|p{22em}|}
\hline
\textbf{测试项目编号:}&dc\_mm/CASE2\\
\hline
\textbf{测试项目描述:}&测试内存分配\\
\hline
\textbf{测试说明:}&随机产生产生块大小,不断申请\\
\hline
\textbf{测试步骤及输入:}& 随机申请大小及数目,包括不合法数据\\
\hline
\textbf{期待结果及输出:}&通过GDB查看内存,查看到指针已正确存储相应的chunk地址,不合法输入返回NULL\\
\hline
\textbf{测试结果:}&无它异常\\
\hline
\textbf{测试所花时间(单位:分钟)}& 约240分钟\\
\hline
\textbf{备注:}&修复计算最佳适应chunk大小中的bug \\
\hline
\end{tabular}
\end{table}

\subsection {内存释放功能测试用例}
\begin{table}[H]
\centering
\begin{tabular}{|p{8em}|p{22em}|}
\hline
\textbf{测试项目编号:}&dc\_mm/CASE3\\
\hline
\textbf{测试项目描述:}&测试内存释放功能\\
\hline
\textbf{测试说明:}&在随机申请的内存里,随机乱序释放相关内存\\
\hline
\textbf{测试步骤及输入:}&随机申请的chunk,调用dc\_free()随机乱序释放\\
\hline
\textbf{期待结果及输出:}&成功释放相关内存,对于不合法的内存释放不予处理\\
\hline
\textbf{测试结果:}&无异常\\
\hline
\textbf{测试所花时间(单位:分钟)}& 约180分钟\\
\hline
\textbf{备注:}& \\
\hline
\end{tabular}
\end{table}


\subsection {内存重分配功能测试用例}
\begin{table}[H]
\centering
\begin{tabular}{|p{8em}|p{22em}|}
\hline
\textbf{测试项目编号:}&dc\_mm/CASE4\\
\hline
\textbf{测试项目描述:}&测试内存重分配功能\\
\hline
\textbf{测试说明:}&在可控范围内,随机申请的内存,释放内存,重新分配内存\\
\hline
\textbf{测试步骤及输入:}&随机申请的chunk,调用dc\_free()随机乱序释放,重新分配内存\\
\hline
\textbf{期待结果及输出:}&通过GDB查看,成功重新分配内存\\
\hline
\textbf{测试结果:}&无异常\\
\hline
\textbf{测试所花时间(单位:分钟)}& 约180分钟\\
\hline
\textbf{备注:}& \\
\hline
\end{tabular}
\end{table}


\subsection {打印统计功能测试用例}
\begin{table}[H]
\centering
\begin{tabular}{|p{8em}|p{22em}|}
\hline
\textbf{测试项目编号:}&dc\_mm/CASE5\\
\hline
\textbf{测试项目描述:}&测试打印统计数据功能\\
\hline
\textbf{测试说明:}&在可控范围内,随机申请的内存,释放内存,打印统计数据\\
\hline
\textbf{测试步骤及输入:}&随机申请的chunk,调用dc\_free()随机乱序释放,并统计相关数据\\
\hline
\textbf{期待结果及输出:}&成功打印统计结果\\
\hline
\textbf{测试结果:}&无异常\\
\hline
\textbf{测试所花时间(单位:分钟)}& 约80分钟\\
\hline
\textbf{备注:}& \\
\hline
\end{tabular}
\end{table}


\subsection{性能及压力测试用例}
\begin{table}[H]
\centering
\begin{tabular}{|p{8em}|p{22em}|}
\hline
\textbf{测试项目编号:}&dc\_mm/CASE6\\
\hline
\textbf{测试项目描述:}&性能压力测试\\
\hline
\textbf{测试说明:}&在内存极限下申请、释放内存\\
\hline
\textbf{测试步骤及输入:}&随机申请大小、数目,随机释放\\
\hline
\textbf{期待结果及输出:}&正确执行\\
\hline
\textbf{测试结果:}& 无异常\\
\hline
\textbf{测试所花时间(单位:分钟)}& 约150分钟\\
\hline
\textbf{备注:}& \\
\hline
\end{tabular}
\end{table}


\subsection{稳定性测试用例}
\begin{table}[H]
\centering
\begin{tabular}{|p{8em}|p{22em}|}
\hline
\textbf{测试项目编号:}&dc\_mm/CASE7\\
\hline
\textbf{测试项目描述:}&稳定性测试\\
\hline
\textbf{测试说明:}&不断申请、释放、重分配内存,以及一些极限条件\\
\hline
\textbf{测试步骤及输入:}&随机数据\\
\hline
\textbf{期待结果及输出:}&内存稳定,无泄漏\\
\hline
\textbf{测试结果:}&连续运行24小时,内存不断变化但一段时间内总是稳定在某一水平\\
\hline
\textbf{测试所花时间(单位:分钟)}& 约60*24分钟\\
\hline
\textbf{备注:}& \\
\hline
\end{tabular}
\end{table}



%
%/***************  END OF cache.tex  **************/
