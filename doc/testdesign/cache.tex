%
%
%/********************************************************************
% * filename:		cache.tex
% * author: 		synckey
% * version:		v1.0
% * datetime:		2011-07-05 13:32:37
% * description:	cache module
% ********************************************************************/
\section{缓存模块测试用例设计}
\subsection{缓存初始化测试用例}

\begin{table}[H]
\centering
\begin{tabular}{|p{8em}|p{22em}|}
\hline
\textbf{测试项目编号:}&dl\_cache/CASE1\\
\hline
\textbf{测试项目描述:}&cache初始化\\
\hline
\textbf{测试说明:}& 测试cache初始化功能,DATrie和ip.list中恢复缓冲\\
\hline
\textbf{测试步骤及输入:}&调用kache\_init()并给定domains.tri和ip.list\\
\hline
\textbf{期待结果及输出:}&通过GDB查看内存,相关内存已正确初始化\\
\hline
\textbf{测试结果:}& 初始化正常\\
\hline
\textbf{测试所花时间(单位:分钟)}& 约130分钟\\
\hline
\textbf{备注:}& \\
\hline
\end{tabular}
\end{table}

\subsection {缓存存取测试用例}
\begin{table}[H]
\centering
\begin{tabular}{|p{8em}|p{22em}|}
\hline
\textbf{测试项目编号:}&dl\_cache/CASE2\\
\hline
\textbf{测试项目描述:}&Cache\_store\\
\hline
\textbf{测试说明:}&随机产生<Key, Data> 对,放入cache中\\
\hline
\textbf{测试步骤及输入:}& 随机数据\\
\hline
\textbf{期待结果及输出:}&通过GDB查看内存,数据已经存入缓冲\\
\hline
\textbf{测试结果:}&正确\\
\hline
\textbf{测试所花时间(单位:分钟)}& 约130分钟\\
\hline
\textbf{备注:}& \\
\hline
\end{tabular}
\end{table}

\subsection {缓存查找测试用例}
\begin{table}[H]
\centering
\begin{tabular}{|p{8em}|p{22em}|}
\hline
\textbf{测试项目编号:}&dl\_cache/CASE3\\
\hline
\textbf{测试项目描述:}&cache retrieve\\
\hline
\textbf{测试说明:}&通过记录随机插入的<key,data>对,对key进行查找,比对data\\
\hline
\textbf{测试步骤及输入:}&记录的<key,data>对\\
\hline
\textbf{期待结果及输出:}&data一致\\
\hline
\textbf{测试结果:}&data一致\\
\hline
\textbf{测试所花时间(单位:分钟)}& 约130分钟\\
\hline
\textbf{备注:}& \\
\hline
\end{tabular}
\end{table}


\subsection{性能及压力测试用例}
\begin{table}[H]
\centering
\begin{tabular}{|p{8em}|p{22em}|}
\hline
\textbf{测试项目编号:}&dl\_cache/CASE4\\
\hline
\textbf{测试项目描述:}&性能压力测试\\
\hline
\textbf{测试说明:}&插入 约180000数据,并全部取出。记录时间\\
\hline
\textbf{测试步骤及输入:}&长度约180000的数据列表\\
\hline
\textbf{期待结果及输出:}&能够在1s内执行完成\\
\hline
\textbf{测试结果:}& <1s\\
\hline
\textbf{测试所花时间(单位:分钟)}& 约130分钟\\
\hline
\textbf{备注:}& \\
\hline
\end{tabular}
\end{table}


\subsection{稳定性测试用例}
\begin{table}[H]
\centering
\begin{tabular}{|p{8em}|p{22em}|}
\hline
\textbf{测试项目编号:}&dl\_cache/CASE5\\
\hline
\textbf{测试项目描述:}&稳定性测试\\
\hline
\textbf{测试说明:}&持续对缓冲存取, 观测内存CPU等资源使用情况\\
\hline
\textbf{测试步骤及输入:}&随机数据\\
\hline
\textbf{期待结果及输出:}&数据匹配,且内存CPU占用稳定\\
\hline
\textbf{测试结果:}&连续运行24小时,内存CPU占用几乎不变\\
\hline
\textbf{测试所花时间(单位:分钟)}& 约24小时\\
\hline
\textbf{备注:}& \\
\hline
\end{tabular}
\end{table}


%
%/***************  END OF cache.tex  **************/
