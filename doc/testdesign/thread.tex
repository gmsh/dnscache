%
%
%/********************************************************************
% * filename:		cache.tex
% * author: 		synckey
% * version:		v1.0
% * datetime:		2011-07-05 13:32:37
% * description:	thread module
% ********************************************************************/
\section{线程模块测试用例设计}
\subsection{DNS查询线程测试用例}

\begin{table}[H]
\centering
\begin{tabular}{|p{8em}|p{22em}|}
\hline
\textbf{测试项目编号:}&thread/CASE1\\
\hline
\textbf{测试项目描述:}&DNS查询线程功能测试\\
\hline
\textbf{测试说明:}& DNS查询线程从工作队列中取出请求,进行DNS查询,将查询结果写入缓存并发给客户端\\
\hline
\textbf{测试步骤及输入:}&工作队列,里面有域名和客户端地址\\
\hline
\textbf{期待结果及输出:}&DNS查询结果正确,成功写入缓存,并发送给客户端,线程之间同步正确\\
\hline
\textbf{测试结果:}& 客户端收到DNS查询结果,本地缓存被更新,线程间同步正确\\
\hline
\textbf{测试所花时间(单位:分钟)}& 约130分钟\\
\hline
\textbf{备注:}& \\
\hline
\end{tabular}
\end{table}

\subsection {TCP工作线程测试用例}
\begin{table}[H]
\centering
\begin{tabular}{|p{8em}|p{22em}|}
\hline
\textbf{测试项目编号:}&thread/CASE2\\
\hline
\textbf{测试项目描述:}&TCP工作线程功能测试\\
\hline
\textbf{测试说明:}&主线程在创建TCP监听套接字后,各个工作线程单独accept客户端请求,
	与客户端交互,查询本地缓存,并把结果返回给客户端\\
\hline
\textbf{测试步骤及输入:}&服务器开启,客户端连接到服务器,请求域名\\
\hline
\textbf{期待结果及输出:}&客户端能够连接到服务器,并收到服务器的正确结果,工作线
程间同步正确\\
\hline
\textbf{测试结果:}&客户端收到服务器发来的响应,多客户端并发访问正常\\
\hline
\textbf{测试所花时间(单位:分钟)}& 约130分钟\\
\hline
\textbf{备注:}& \\
\hline
\end{tabular}
\end{table}

\subsection {UDP服务器线程测试用例}
\begin{table}[H]
\centering
\begin{tabular}{|p{8em}|p{22em}|}
\hline
\textbf{测试项目编号:}&thread/CASE3\\
\hline
\textbf{测试项目描述:}&UDP服务器功能测试\\
\hline
\textbf{测试说明:}&接收客户端的UDP请求,查询本地缓存,返回结果\\
\hline
\textbf{测试步骤及输入:}&开启UDP服务器,客户端用UDP连接到服务器,请求域名\\
\hline
\textbf{期待结果及输出:}&服务器能接收到客户端的请求,并把结果返回给客户端\\
\hline
\textbf{测试结果:}&服务器迭代处理客户端的请求,客户端收到正确结果\\
\hline
\textbf{测试所花时间(单位:分钟)}& 约130分钟\\
\hline
\textbf{备注:}& \\
\hline
\end{tabular}
\end{table}


\subsection{性能及压力测试用例}
\begin{table}[H]
\centering
\begin{tabular}{|p{8em}|p{22em}|}
\hline
\textbf{测试项目编号:}&thread/CASE4\\
\hline
\textbf{测试项目描述:}&性能压力测试\\
\hline
\textbf{测试说明:}&查询1843314个域名,多线程并发访问DNS服务器,获取IP地址。记录时间\\
\hline
\textbf{测试步骤及输入:}&客户端请求1843314个域名\\
\hline
\textbf{期待结果及输出:}&服务器能够获得正确的IP地址,并返回给客户端\\
\hline
\textbf{测试结果:}& 在双核处理器,2G内存的32位机器上,开启200个DNS查询线程,耗时
2h26min查询结束,客户端收到正确结果。\\
\hline
\textbf{测试所花时间(单位:分钟)}& 约130分钟\\
\hline
\textbf{备注:}& \\
\hline
\end{tabular}
\end{table}


\subsection{稳定性测试用例}
\begin{table}[H]
\centering
\begin{tabular}{|p{8em}|p{22em}|}
\hline
\textbf{测试项目编号:}&thread/CASE5\\
\hline
\textbf{测试项目描述:}&稳定性测试\\
\hline
\textbf{测试说明:}&多台主机,多线程,同时向服务器请求大量域名\\
\hline
\textbf{测试步骤及输入:}&每个线程从1883314个域名中随机挑选1-200000万不等的域名,
向服务器请求\\
\hline
\textbf{期待结果及输出:}&服务器能正确工作\\
\hline
\textbf{测试结果:}&在服务器端的输入速度为11M/s的情况下,客户端仍能收到正确应答。\\
\hline
\textbf{测试所花时间(单位:分钟)}& 约130分钟\\
\hline
\textbf{备注:}& \\
\hline
\end{tabular}
\end{table}



%
%/***************  END OF cache.tex  **************/
