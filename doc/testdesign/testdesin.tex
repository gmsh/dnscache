%
%
%/********************************************************************
% * filename:           testdesin.tex
% * author:             synckey
% * version:            v1.0
% * datetime:           2011-07-05 08:33:29
% * description:        testdesing
% ********************************************************************/

\documentclass[12pt, a4paper, titlepage]{article}
\usepackage{titletoc}
\usepackage{fontspec}
\usepackage{graphicx}
\usepackage{array}
\usepackage{longtable}
\usepackage{paralist}
\usepackage{float,longtable}
\usepackage{enumerate}
\usepackage{xcolor,listings}
\usepackage{booktabs, multirow}
\usepackage[pagestyles]{titlesec}
\usepackage{xeCJK}
\lstset{
	numbers=left,%行号
	framexleftmargin = 2em,%背景框
	frame = none,
	backgroundcolor = \color[RGB]{245, 245,244},
	keywordstyle = \bf\color{blue},
	identifierstyle = \bf,
	numberstyle=\color[RGB]{0,192,192},
	commentstyle=\it\color[RGB]{0,96,96},
	stringstyle=\rmfamily\slshape\color[RGB]{128,0,0},
	showstringspaces=false
}


\newcommand{\codebox}[3]
{
\begin{center}
\fbox{
	\begin{minipage}[b]{130mm}
		#1\\
		#2
		\begin{flushright}
		返回值:#3
		\end{flushright}
	\end{minipage}
}
\end{center}
}


\titlecontents{section}[4em]{\small}{\contentslabel{3.3em}}{}
	{\titlerule*[0.5pc]{$\cdot$}\contentspage}
\titlecontents{subsection}[7em]{\small}{\contentslabel{3.3em}}{}
	{\titlerule*[0.5pc]{$\cdot$}\contentspage}
\titlecontents{subsubsection}[7em]{\small}{\contentslabel{3.3em}}{}
	{\titlerule*[0.5pc]{$\cdot$}\contentspage}
\renewcommand{\today}{\number\year-\number\month-\number\day}
\setmainfont[BoldFont=SimHei]{SimSun}
\setmonofont{SimSun}
\XeTeXlinebreaklocale"zh"
\title{DNS缓存系统测试设计}
\author{v1.0}
\date{2011-07-05}
\newpagestyle{mypagestyle}{
	\sethead{\sectiontitle}{}{$\cdot$~\thepage~$\cdot$
	}
	\setheadrule{1pt}
	\setfoot{}{}{\headrule}
}
\newpagestyle{myempty}{
	\sethead{\sectiontitle}{}{}
	\setheadrule{1pt}
	\setfoot{}{}{\headrule}
}
\renewcommand{\contentsname}{目\quad 录}
\renewcommand{\figurename}{图}
\renewcommand{\tablename}{表}
\renewcommand{\refname}{参考文献}
\pagestyle{mypagestyle}
\makeatletter
\let\@afterindentfalse\@afterindenttrue
\@afterindenttrue
\makeatother%

\setlength{\parindent}{2em}%中文缩进两个汉字位
\linespread{1.25}
\usepackage[pagebackref,colorlinks, linkcolor = blue , urlcolor = blue]{hyperref}
\newcommand{\tabincell}[2]{\begin{tabular}{@{}#1@{}}#2\end{tabular}}

\begin{document}
\maketitle


\newpage
\pagenumbering{Roman}%


\tableofcontents
\newpage
\pagenumbering{arabic}

\section{项目描述}
\subsection{项目总述}
\begin{table}[H]
\centering
%\centering\Large\textbf{项目描述部分}\\[15pt]
\begin{tabular}{|p{5em}|p{25em}|}
\hline
\textbf{项目名称:}&DNS缓存系统\\
\hline
\textbf{项目提交人:}&李琳,郭茂盛,王健,王军委\\
\hline
\textbf{提交时间:}& 2011-07-12 17时\\
\hline
\textbf{研发建议完成时间:}&2011-07-10 17时\\
\hline
\textbf{项目背景描述:}&{在百度的大规模集群环境中,有这样一个模块,在它工作时,需
要根据用户指定的机器名列表,与这些机器建立连接完成一定任务。
在这个过程中,会大量进行域名的解析,不仅对DNS 服务器造成较大压力,还容易发生解析
失败的情况,设计一个系统,来应对突发性的、海量的DNS 请求并在超出能力极限的情况下
自我保护。该系统同时可以应用于短连接大并发的DNS 请求并在超出能力极限的情况下自我
保护。
本项目设计的系统为其它系统服务,根据客户提出的域名,返回域名对应的IP 地址。用户
可以一次提出一个请求,也可以在一个请求中请求若干域名的IP地址。系统根据用户的请求
,从DNS 服务器获得域名和IP 地址的对应关系,并进行缓存,从而在用户下次请求时,可
以快速响应。当服务器的缓放满之后,就使用LRU算法,对缓冲区进行更新。}\\
\hline
\textbf{测试要求:}& 1.对系统设计测试用例,包括:新功能测试用例、基本功能回归测
试用例、性能及压力测试用例、稳定性测试用例。
2.根据测试用例对系统进行单元测试、集成测试。对系统的功能、性能、压力、稳定性进
行测试。
3.发现缺陷并及时修复\\
\hline
\textbf{其他:}& 记录测试中出现的缺陷的原因、位置、修复状况。\\
\hline
\end{tabular}
\end{table}


%
%
%/********************************************************************
% * filename:		cache.tex
% * author: 		synckey
% * version:		v1.0
% * datetime:		2011-07-05 13:32:37
% * description:	cache module
% ********************************************************************/
\section{缓存模块测试用例设计}
\subsection{缓存初始化测试用例}

\begin{table}[H]
\centering
\begin{tabular}{|p{8em}|p{22em}|}
\hline
\textbf{测试项目编号:}&dl\_cache/CASE1\\
\hline
\textbf{测试项目描述:}&cache初始化\\
\hline
\textbf{测试说明:}& 测试cache初始化功能,DATrie和ip.list中恢复缓冲\\
\hline
\textbf{测试步骤及输入:}&调用kache\_init()并给定domains.tri和ip.list\\
\hline
\textbf{期待结果及输出:}&通过GDB查看内存,相关内存已正确初始化\\
\hline
\textbf{测试结果:}& 初始化正常\\
\hline
\textbf{测试所花时间(单位:分钟)}& 约130分钟\\
\hline
\textbf{备注:}& \\
\hline
\end{tabular}
\end{table}

\subsection {缓存存取测试用例}
\begin{table}[H]
\centering
\begin{tabular}{|p{8em}|p{22em}|}
\hline
\textbf{测试项目编号:}&dl\_cache/CASE2\\
\hline
\textbf{测试项目描述:}&Cache\_store\\
\hline
\textbf{测试说明:}&随机产生<Key, Data> 对,放入cache中\\
\hline
\textbf{测试步骤及输入:}& 随机数据\\
\hline
\textbf{期待结果及输出:}&通过GDB查看内存,数据已经存入缓冲\\
\hline
\textbf{测试结果:}&正确\\
\hline
\textbf{测试所花时间(单位:分钟)}& 约130分钟\\
\hline
\textbf{备注:}& \\
\hline
\end{tabular}
\end{table}

\subsection {缓存查找测试用例}
\begin{table}[H]
\centering
\begin{tabular}{|p{8em}|p{22em}|}
\hline
\textbf{测试项目编号:}&dl\_cache/CASE3\\
\hline
\textbf{测试项目描述:}&cache retrieve\\
\hline
\textbf{测试说明:}&通过记录随机插入的<key,data>对,对key进行查找,比对data\\
\hline
\textbf{测试步骤及输入:}&记录的<key,data>对\\
\hline
\textbf{期待结果及输出:}&data一致\\
\hline
\textbf{测试结果:}&data一致\\
\hline
\textbf{测试所花时间(单位:分钟)}& 约130分钟\\
\hline
\textbf{备注:}& \\
\hline
\end{tabular}
\end{table}


\subsection{性能及压力测试用例}
\begin{table}[H]
\centering
\begin{tabular}{|p{8em}|p{22em}|}
\hline
\textbf{测试项目编号:}&dl\_cache/CASE4\\
\hline
\textbf{测试项目描述:}&性能压力测试\\
\hline
\textbf{测试说明:}&插入 约180000数据,并全部取出。记录时间\\
\hline
\textbf{测试步骤及输入:}&长度约180000的数据列表\\
\hline
\textbf{期待结果及输出:}&能够在1s内执行完成\\
\hline
\textbf{测试结果:}& <1s\\
\hline
\textbf{测试所花时间(单位:分钟)}& 约130分钟\\
\hline
\textbf{备注:}& \\
\hline
\end{tabular}
\end{table}


\subsection{稳定性测试用例}
\begin{table}[H]
\centering
\begin{tabular}{|p{8em}|p{22em}|}
\hline
\textbf{测试项目编号:}&dl\_cache/CASE5\\
\hline
\textbf{测试项目描述:}&稳定性测试\\
\hline
\textbf{测试说明:}&持续对缓冲存取, 观测内存CPU等资源使用情况\\
\hline
\textbf{测试步骤及输入:}&随机数据\\
\hline
\textbf{期待结果及输出:}&数据匹配,且内存CPU占用稳定\\
\hline
\textbf{测试结果:}&连续运行24小时,内存CPU占用几乎不变\\
\hline
\textbf{测试所花时间(单位:分钟)}& 约24小时\\
\hline
\textbf{备注:}& \\
\hline
\end{tabular}
\end{table}


%
%/***************  END OF cache.tex  **************/

%
%
%/********************************************************************
% * filename:		cache.tex
% * author: 		synckey
% * version:		v1.0
% * datetime:		2011-07-05 13:32:37
% * description:	thread module
% ********************************************************************/
\section{线程模块测试用例设计}
\subsection{DNS查询线程测试用例}

\begin{table}[H]
\centering
\begin{tabular}{|p{8em}|p{22em}|}
\hline
\textbf{测试项目编号:}&thread/CASE1\\
\hline
\textbf{测试项目描述:}&DNS查询线程功能测试\\
\hline
\textbf{测试说明:}& DNS查询线程从工作队列中取出请求,进行DNS查询,将查询结果写入缓存并发给客户端\\
\hline
\textbf{测试步骤及输入:}&工作队列,里面有域名和客户端地址\\
\hline
\textbf{期待结果及输出:}&DNS查询结果正确,成功写入缓存,并发送给客户端,线程之间同步正确\\
\hline
\textbf{测试结果:}& 客户端收到DNS查询结果,本地缓存被更新,线程间同步正确\\
\hline
\textbf{测试所花时间(单位:分钟)}& 约130分钟\\
\hline
\textbf{备注:}& \\
\hline
\end{tabular}
\end{table}

\subsection {TCP工作线程测试用例}
\begin{table}[H]
\centering
\begin{tabular}{|p{8em}|p{22em}|}
\hline
\textbf{测试项目编号:}&thread/CASE2\\
\hline
\textbf{测试项目描述:}&TCP工作线程功能测试\\
\hline
\textbf{测试说明:}&主线程在创建TCP监听套接字后,各个工作线程单独accept客户端请求,
	与客户端交互,查询本地缓存,并把结果返回给客户端\\
\hline
\textbf{测试步骤及输入:}&服务器开启,客户端连接到服务器,请求域名\\
\hline
\textbf{期待结果及输出:}&客户端能够连接到服务器,并收到服务器的正确结果,工作线
程间同步正确\\
\hline
\textbf{测试结果:}&客户端收到服务器发来的响应,多客户端并发访问正常\\
\hline
\textbf{测试所花时间(单位:分钟)}& 约130分钟\\
\hline
\textbf{备注:}& \\
\hline
\end{tabular}
\end{table}

\subsection {UDP服务器线程测试用例}
\begin{table}[H]
\centering
\begin{tabular}{|p{8em}|p{22em}|}
\hline
\textbf{测试项目编号:}&thread/CASE3\\
\hline
\textbf{测试项目描述:}&UDP服务器功能测试\\
\hline
\textbf{测试说明:}&接收客户端的UDP请求,查询本地缓存,返回结果\\
\hline
\textbf{测试步骤及输入:}&开启UDP服务器,客户端用UDP连接到服务器,请求域名\\
\hline
\textbf{期待结果及输出:}&服务器能接收到客户端的请求,并把结果返回给客户端\\
\hline
\textbf{测试结果:}&服务器迭代处理客户端的请求,客户端收到正确结果\\
\hline
\textbf{测试所花时间(单位:分钟)}& 约130分钟\\
\hline
\textbf{备注:}& \\
\hline
\end{tabular}
\end{table}


\subsection{性能及压力测试用例}
\begin{table}[H]
\centering
\begin{tabular}{|p{8em}|p{22em}|}
\hline
\textbf{测试项目编号:}&thread/CASE4\\
\hline
\textbf{测试项目描述:}&性能压力测试\\
\hline
\textbf{测试说明:}&查询1843314个域名,多线程并发访问DNS服务器,获取IP地址。记录时间\\
\hline
\textbf{测试步骤及输入:}&客户端请求1843314个域名\\
\hline
\textbf{期待结果及输出:}&服务器能够获得正确的IP地址,并返回给客户端\\
\hline
\textbf{测试结果:}& 在双核处理器,2G内存的32位机器上,开启200个DNS查询线程,耗时
2h26min查询结束,客户端收到正确结果。\\
\hline
\textbf{测试所花时间(单位:分钟)}& 约130分钟\\
\hline
\textbf{备注:}& \\
\hline
\end{tabular}
\end{table}


\subsection{稳定性测试用例}
\begin{table}[H]
\centering
\begin{tabular}{|p{8em}|p{22em}|}
\hline
\textbf{测试项目编号:}&thread/CASE5\\
\hline
\textbf{测试项目描述:}&稳定性测试\\
\hline
\textbf{测试说明:}&多台主机,多线程,同时向服务器请求大量域名\\
\hline
\textbf{测试步骤及输入:}&每个线程从1883314个域名中随机挑选1-200000万不等的域名,
向服务器请求\\
\hline
\textbf{期待结果及输出:}&服务器能正确工作\\
\hline
\textbf{测试结果:}&在服务器端的输入速度为11M/s的情况下,客户端仍能收到正确应答。\\
\hline
\textbf{测试所花时间(单位:分钟)}& 约130分钟\\
\hline
\textbf{备注:}& \\
\hline
\end{tabular}
\end{table}



%
%/***************  END OF cache.tex  **************/

%
%
%/********************************************************************
% * filename:		cache.tex
% * author: 		synckey
% * version:		v1.0
% * datetime:		2011-07-05 13:32:37
% * description:	cache module
% ********************************************************************/
\section{内存管理模块测试用例设计}
\subsection{内存管理初始化功能测试用例}

\begin{table}[H]
\centering
\begin{tabular}{|p{8em}|p{22em}|}
\hline
\textbf{测试项目编号:}&dc\_mm/CASE1\\
\hline
\textbf{测试项目描述:}&测试内存管理模块初始化\\
\hline
\textbf{测试说明:}& 测试dc\_mm初始化功能,预分配特定大小的chunk各若干\\
\hline
\textbf{测试步骤及输入:}&配置随机的chunks大小以及数目,调用mm\_init(),无输入\\
\hline
\textbf{期待结果及输出:}&通过GDB查看内存,相关chunk已经正确初始化\\
\hline
\textbf{测试结果:}& 初始化正常\\
\hline
\textbf{测试所花时间(单位:分钟)}& 约160分钟\\
\hline
\textbf{备注:}& \\
\hline
\end{tabular}
\end{table}

\subsection {内存分配功能测试用例}
\begin{table}[H]
\centering
\begin{tabular}{|p{8em}|p{22em}|}
\hline
\textbf{测试项目编号:}&dc\_mm/CASE2\\
\hline
\textbf{测试项目描述:}&测试内存分配\\
\hline
\textbf{测试说明:}&随机产生产生块大小,不断申请\\
\hline
\textbf{测试步骤及输入:}& 随机申请大小及数目,包括不合法数据\\
\hline
\textbf{期待结果及输出:}&通过GDB查看内存,查看到指针已正确存储相应的chunk地址,不合法输入返回NULL\\
\hline
\textbf{测试结果:}&无它异常\\
\hline
\textbf{测试所花时间(单位:分钟)}& 约240分钟\\
\hline
\textbf{备注:}&修复计算最佳适应chunk大小中的bug \\
\hline
\end{tabular}
\end{table}

\subsection {内存释放功能测试用例}
\begin{table}[H]
\centering
\begin{tabular}{|p{8em}|p{22em}|}
\hline
\textbf{测试项目编号:}&dc\_mm/CASE3\\
\hline
\textbf{测试项目描述:}&测试内存释放功能\\
\hline
\textbf{测试说明:}&在随机申请的内存里,随机乱序释放相关内存\\
\hline
\textbf{测试步骤及输入:}&随机申请的chunk,调用dc\_free()随机乱序释放\\
\hline
\textbf{期待结果及输出:}&成功释放相关内存,对于不合法的内存释放不予处理\\
\hline
\textbf{测试结果:}&无异常\\
\hline
\textbf{测试所花时间(单位:分钟)}& 约180分钟\\
\hline
\textbf{备注:}& \\
\hline
\end{tabular}
\end{table}


\subsection {内存重分配功能测试用例}
\begin{table}[H]
\centering
\begin{tabular}{|p{8em}|p{22em}|}
\hline
\textbf{测试项目编号:}&dc\_mm/CASE4\\
\hline
\textbf{测试项目描述:}&测试内存重分配功能\\
\hline
\textbf{测试说明:}&在可控范围内,随机申请的内存,释放内存,重新分配内存\\
\hline
\textbf{测试步骤及输入:}&随机申请的chunk,调用dc\_free()随机乱序释放,重新分配内存\\
\hline
\textbf{期待结果及输出:}&通过GDB查看,成功重新分配内存\\
\hline
\textbf{测试结果:}&无异常\\
\hline
\textbf{测试所花时间(单位:分钟)}& 约180分钟\\
\hline
\textbf{备注:}& \\
\hline
\end{tabular}
\end{table}


\subsection {打印统计功能测试用例}
\begin{table}[H]
\centering
\begin{tabular}{|p{8em}|p{22em}|}
\hline
\textbf{测试项目编号:}&dc\_mm/CASE5\\
\hline
\textbf{测试项目描述:}&测试打印统计数据功能\\
\hline
\textbf{测试说明:}&在可控范围内,随机申请的内存,释放内存,打印统计数据\\
\hline
\textbf{测试步骤及输入:}&随机申请的chunk,调用dc\_free()随机乱序释放,并统计相关数据\\
\hline
\textbf{期待结果及输出:}&成功打印统计结果\\
\hline
\textbf{测试结果:}&无异常\\
\hline
\textbf{测试所花时间(单位:分钟)}& 约80分钟\\
\hline
\textbf{备注:}& \\
\hline
\end{tabular}
\end{table}


\subsection{性能及压力测试用例}
\begin{table}[H]
\centering
\begin{tabular}{|p{8em}|p{22em}|}
\hline
\textbf{测试项目编号:}&dc\_mm/CASE6\\
\hline
\textbf{测试项目描述:}&性能压力测试\\
\hline
\textbf{测试说明:}&在内存极限下申请、释放内存\\
\hline
\textbf{测试步骤及输入:}&随机申请大小、数目,随机释放\\
\hline
\textbf{期待结果及输出:}&正确执行\\
\hline
\textbf{测试结果:}& 无异常\\
\hline
\textbf{测试所花时间(单位:分钟)}& 约150分钟\\
\hline
\textbf{备注:}& \\
\hline
\end{tabular}
\end{table}


\subsection{稳定性测试用例}
\begin{table}[H]
\centering
\begin{tabular}{|p{8em}|p{22em}|}
\hline
\textbf{测试项目编号:}&dc\_mm/CASE7\\
\hline
\textbf{测试项目描述:}&稳定性测试\\
\hline
\textbf{测试说明:}&不断申请、释放、重分配内存,以及一些极限条件\\
\hline
\textbf{测试步骤及输入:}&随机数据\\
\hline
\textbf{期待结果及输出:}&内存稳定,无泄漏\\
\hline
\textbf{测试结果:}&连续运行24小时,内存不断变化但一段时间内总是稳定在某一水平\\
\hline
\textbf{测试所花时间(单位:分钟)}& 约60*24分钟\\
\hline
\textbf{备注:}& \\
\hline
\end{tabular}
\end{table}



%
%/***************  END OF cache.tex  **************/



\end{document}
%
%/***************  END OF testdesin.tex  **************/
