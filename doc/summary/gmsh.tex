%
%
%/********************************************************************
% * filename:           gmsh.tex
% * author:             wakemecn
% * version:            v1.0
% * datetime:           Jun 7th, 2011 
% ********************************************************************/

\documentclass[12pt, a4paper, titlepage]{article}
\usepackage{titletoc}
\usepackage{fontspec}
\usepackage{graphicx}
\usepackage{array}
\usepackage{longtable}
\usepackage{paralist}
\usepackage{float,longtable}
\usepackage{enumerate}
\usepackage{xcolor,listings}
\usepackage{booktabs, multirow}
\usepackage[pagestyles]{titlesec}
\usepackage{xeCJK}


\titlecontents{section}[4em]{\small}{\contentslabel{3.3em}}{}
	{\titlerule*[0.5pc]{$\cdot$}\contentspage}
\titlecontents{subsection}[7em]{\small}{\contentslabel{3.3em}}{}
	{\titlerule*[0.5pc]{$\cdot$}\contentspage}
\titlecontents{subsubsection}[7em]{\small}{\contentslabel{3.3em}}{}
	{\titlerule*[0.5pc]{$\cdot$}\contentspage}
\renewcommand{\today}{\number\year-\number\month-\number\day}
\setmainfont[BoldFont=SimHei]{SimSun}
\setmonofont{SimSun}
\XeTeXlinebreaklocale"zh"

\newpagestyle{mypagestyle}{
	\sethead{\sectiontitle}{}{$\cdot$~\thepage~$\cdot$
	}
	\setheadrule{1pt}
	\setfoot{}{}{\headrule}
}
\newpagestyle{myempty}{
	\sethead{\sectiontitle}{}{}
	\setheadrule{1pt}
	\setfoot{}{}{\headrule}
}
\renewcommand{\contentsname}{目\quad 录}
\renewcommand{\figurename}{图}
\renewcommand{\tablename}{表}
\renewcommand{\refname}{参考文献}
\pagestyle{mypagestyle}
\makeatletter
\let\@afterindentfalse\@afterindenttrue
\@afterindenttrue
\makeatother%

\usepackage{ulem}
\setlength{\parindent}{2em}%中文缩进两个汉字位
\linespread{1.25}
\usepackage[pagebackref,colorlinks, linkcolor = blue , urlcolor = blue]{hyperref}
\newcommand{\tabincell}[2]{\begin{tabular}{@{}#1@{}}#2\end{tabular}}



%\vspace{26mm}
\begin{document}
%\maketitle
\begin{titlepage}
\centering
\Huge\textbf{软件工程实训总结报告}\\[7pt]
\large{通用DNS缓存设计——百度}\\[50mm]

%\begin{flushcenter}
\textbf{作 \qquad 者} {  郭茂盛}
%\end{flushcenter}

\end{titlepage}

\newpage
\pagenumbering{arabic}
\par{
	时间过的真快转眼间我期望已久的实训周已经结束。经过两周的实训练习让我学到了许多
	知识回头想想实训这几天我确实是有很大收获的。在老师的耐心指导和鼓励下,圆满完成
	了实习任务,从总体上达到了实习 预期的目标和要求。这次总实习 给了我一次全面的、
	系统的实践锻炼的机会,巩固了所学的理论知识,增强了我的实际操作能力,我进一步从
	实践中认识到编写代码对于我们软件工程专业的重要性。	
}

\par{
	我以后在工作中光有理论知识是不够的,还要能把理论运用到实践中去才行。理论知识
	固然重要,可是无实践的理论就是空谈。真正做到理论与实践的相结合,将理论真正用
	到实践中去,才能更好的将自己的才华展现出来。我以前总以为看书看的明白,也理解
	就得了,经过这次的实训,我现在终于明白,没有实践所学的东西就不属于你的。俗话
	说:“尽信书则不如无书”我们要读好书,而不是读死书。实训期间,让我学到了很多东
	西,不仅使我在理论上对网络管理有了全新的认识,在实践能力上也得到了提高,真正
	地做到了学以致用,更学到了很多做人的道理,对我来说受益非浅。除此以外,我还学
	会了如何更好地与别人沟通,如何更好地去陈述自己的观点,如何说服别人认同自己的
	观点。第一次亲身感受到理论与实际的相结合,让我大开眼界。也是对以前所学知识的
	一个初审吧!这次实训对于我以后学习、找工作也真是受益菲浅,在短短的两周中相信
	这些宝贵的经验会成为我今后成功的重要的基石。	
}

\par{
	我感受最深的,还有以下几点:
}
\begin{description}
	\item[其一]实训是对每个人综合能力的检验。要想做好任何事,除了自己平时要有一定
	的功底外,我们还需要一定的实践动手能力,操作能力。
	\item[其二]此次实训,我深深体会到了积累知识的重要性。俗话说:“要想为事业多添
	一把火,自己就得多添一捆材”。我对此话深有感触。
	\item[再次]“纸上得来终觉浅,绝知此事要躬行!”在短暂的实习过程中,让我深深的感
	觉到自己在实际运用中的专业知识的匮乏,刚开始的一段时间里,对一些工作感到无从
	下手,茫然不知所措,这让我感到非常的难过。在学校总以为自己学的不错,一旦接触
	到实际,才发现自己知道的是多么少,这时才真正领悟到“学无止境”的含义。这也许是
	我一个人的感觉。
\end{description}

\end{document}
%
%/***************  END OF testdesin.tex  **************/
