%
%
%/********************************************************************
% * filename:           wakemecn.tex
% * author:             wakemecn
% * version:            v1.0
% * datetime:           Jun 7th, 2011 
% ********************************************************************/

\documentclass[12pt, a4paper, titlepage]{article}
\usepackage{titletoc}
\usepackage{fontspec}
\usepackage{graphicx}
\usepackage{array}
\usepackage{longtable}
\usepackage{paralist}
\usepackage{float,longtable}
\usepackage{enumerate}
\usepackage{xcolor,listings}
\usepackage{booktabs, multirow}
\usepackage[pagestyles]{titlesec}
\usepackage{xeCJK}


\titlecontents{section}[4em]{\small}{\contentslabel{3.3em}}{}
	{\titlerule*[0.5pc]{$\cdot$}\contentspage}
\titlecontents{subsection}[7em]{\small}{\contentslabel{3.3em}}{}
	{\titlerule*[0.5pc]{$\cdot$}\contentspage}
\titlecontents{subsubsection}[7em]{\small}{\contentslabel{3.3em}}{}
	{\titlerule*[0.5pc]{$\cdot$}\contentspage}
\renewcommand{\today}{\number\year-\number\month-\number\day}
\setmainfont[BoldFont=SimHei]{SimSun}
\setmonofont{SimSun}
\XeTeXlinebreaklocale"zh"

\newpagestyle{mypagestyle}{
	\sethead{\sectiontitle}{}{$\cdot$~\thepage~$\cdot$
	}
	\setheadrule{1pt}
	\setfoot{}{}{\headrule}
}
\newpagestyle{myempty}{
	\sethead{\sectiontitle}{}{}
	\setheadrule{1pt}
	\setfoot{}{}{\headrule}
}
\renewcommand{\contentsname}{目\quad 录}
\renewcommand{\figurename}{图}
\renewcommand{\tablename}{表}
\renewcommand{\refname}{参考文献}
\pagestyle{mypagestyle}
\makeatletter
\let\@afterindentfalse\@afterindenttrue
\@afterindenttrue
\makeatother%

\usepackage{ulem}
\setlength{\parindent}{2em}%中文缩进两个汉字位
\linespread{1.25}
\usepackage[pagebackref,colorlinks, linkcolor = blue , urlcolor = blue]{hyperref}
\newcommand{\tabincell}[2]{\begin{tabular}{@{}#1@{}}#2\end{tabular}}

%\vspace{26mm}
\begin{document}
%\maketitle
\begin{titlepage}
\centering
\Huge\textbf{软件工程实训总结报告}\\[7pt]
\large{通用DNS缓存设计——百度}\\[50mm]

%\begin{flushcenter}
\rule{35mm}{0pt}
\textbf{作 \qquad 者} { \hfill 王军委  \quad\quad \quad \quad \quad \quad} \\
\rule{35mm}{0pt}
\textbf{学 \qquad 号} { \hfill   200800300237 \quad\quad \quad \quad \quad}
%\end{flushcenter}

\end{titlepage}

\newpage
\pagenumbering{Roman}%
\tableofcontents

\newpage
\pagenumbering{arabic}
\section{软件工程实训}
\subsection{软件工程实训目标}
\begin{asparaenum}
\item{了解企业的文化和制度,熟悉企业的工作流程和工作方式。}
\item{熟悉实际项目分析、设计、开发、测试、提交等完整流程,熟悉企业各类文档模板,
并按照这些模板撰写项目文档。}
\item{悉使用各种开发工具、系统设计工具、项目管理工具和缺陷管理工具;熟练使用常用
服务器软件安装、配置、开发。}
\item{掌握开发架构,能独立设计完成企业中小型解决方案。}
\item{形成良好的编码习惯,熟练掌握如何使代码更加易于理解和可读。}
\item{养成良好的表达、沟通和团队协作能力,掌握快速学习方法,培养良好的分析问题
和解决问题能力。}
\end{asparaenum}

\section{软件项目概述}
\subsection{软件项目名称}
通用DNS缓存系统设计。

\subsection{软件项目要求}
\paragraph{场景描述:\\[7pt]}  

在百度的大规模集群环境中,有这样一个模块,在它工作时,需要根据用户指定的机器名列
表,与这些机器建立连接完成一定任务。在这个过程中,会大量进行域名的解析,不仅对
DNS服务器造成较大压力,还容易发生解析失败的情况,请设计一个系统,来解决这个问题
。(该系统还可以应用于短连接大并发的系统模型中)。

\paragraph{功能需求:}
\begin{asparaenum}
\item{可配置的缓存更新策略,如缓存的老化周期,缓存的大小。根据LRU来实现缓存更新
算法,并可扩展。}
\item{处理网络请求,请求内容为一个或者n个主机名,返回对应的ip地址。包格式请考虑
以后的扩展性。}
\item{利用多核CPU,采用多线程实现。}
\end{asparaenum}
\paragraph{实现要求:}
\begin{asparaenum}
\item{文档齐全:设计文档、单元测试文档、自动测试文档。文档的撰写思路明晰,
内容完善。}
\item{设计方面考虑完整,模块划分清晰。}
\item{使用C/C++实现。实现功能完整,代码符合代码规范,好的代码的可读性和可维护性
,必要的注释。}
\item{使用GTest或CPPUnit进行单元测试。}
\item{完善的自动化测试用例。撰写程序对整体功能进行测试。}
\end{asparaenum}

\subsection{软件特点}
本系统为其它系统服务,根据客户提出的域名,返回域名对应的IP 地址。用户可以一次提
出一个请求,也可以在一个请求中请求若干域名的IP地址。系统根据用户的请求,从DNS 服
务器获得域名和IP 地址的对应关系,并进行缓存,从而在用户下次请求时,可以快速响应
。当服务器的缓放满之后,就使用LIRS 算法,对缓冲区进行更新。

\section{技术技能}
本次项目实训主要用到了C语言编程和网络编程的知识。

\section{系统设计}
\subsection{整体框架}
\par{DNS缓存系统是一个DNS的一个高速缓存服务器,提供DNS缓存服务。传统的DNS服务器
	一次只能查询一个域名,而且响应速度慢。本系统将为客户端代理DNS请求,把DNS查
	询结果返回给客户端,并且把结果缓存在本地。当客户下一次请求同样的域名时,就
	可以在本地的缓存中查找,如果缓存中存在未被替换的结果,则返回该结果;否则,系
	统重新请求DNS服务器,并将DNS服务器返回结果存入缓存中。客户可以在一次请求中
	查询多个域名的IP地址,本系统予以响应。我们把整个系统分为四个大的模块:	
}
\begin{itemize}
     \item{缓存管理模块}
	\item{内存管理模块}
     \item{线程池管理模块}
     \item{测试客户端}
\end{itemize}

\section{个人贡献}
	本人在本次项目实训中主要参与了内存管理模块、测试客户端代码的编写,内存管理
	模块、通用接口的测试,集成测试,性能测试,以及文档的编写。

\section{团队收获}
经过近一月的实训,我们的小组每个人都收获颇多。不仅个人素质、专
业能力得到锻炼和提升,同时对如何更有效的进行团队合作方式有了更深刻的理解。
\subsection{软件工程开发方法}
\par{在实训期间,我们小组将软件工程开发理论方法用于实际,切实体会到软件工程开发方
	法在项目开发中起到的重要作用,从需求分析到系统设计,从代码开发到系统测试,每
	一步都是在小组成员的讨论商议中完成。	
}
\par{在需求分析阶段,我们通过题目要求,着重分析系统的性能要求,实现在 8*2.33G CPU,
	16G 内存,64 位主机上处理速度需要达到 5000req/s。此数值将根据具体主机的不同而
	调整,基本评判标准是在相同缓存策略下,处理速度越快越好。
}
\par{为了实现需求中的各种功能、性能要求,我们团队在系统设计阶段对各种算法、数据结构
	等方面做了折中,选取代价最小而性能最高的方法设计系统。在UDP和TCP协议的选择中,
	我们做出这样初步设计为系统提供两个服务器,一个 UDP 服务器,和一个 TCP 服务器。
	客户端根据它的报文大小来选择使用 UDP 还是 TCP。在替换算法方面,LIRS 的效率比 
	LRU 高,但是实现的代价比 LRU高不了多少,我们选择使用 LIRS。在数据结构方面,我
	们选择了冲突更小的字典树替代了冲突较高的HASH函数。为了进一步提高系统性能,我们
	采用一次请求两次返回机制。详见系统设计说明书。
}
\par{在代码开发阶段。小组采用迭代式开发方法,每完成一个子模块就进行测试,保证了后续
	开发的正确性及稳定性。	
}
\par{在最后的测试阶段,我们应用代码走差、单元测试、集成测试、功能测试等测试方法,
	设计测试用例对系统的功能、性能、安全性进行测试,并形成测试报告。	
}
\par{总之,通过本次实训平台,给我们提高了一个良好的实践机会,将所学理论知识进一
	步用于实践中,在实践中巩固和检验理论。	
}

\subsection{小组内团队合作}
小组内成员在项目开发过程中通过不断调整工作分配方式和问题解决办法,循序渐进中做到
充分发挥每个人的优势,将每个人的知识、能力和经验用于团队开发中。小组里的每个角色
都代表了对项目的一种独一无二的观点,但是没有哪个个人能够完全代表所有的不同质量目
标。
\subsubsection{组成员资源共享}
\par{为了方便共享资源和代码,了解他人进度。小组集各个成员的智慧于一体,通过多种
方式实现资源管理与共享,建立ftp、快盘账户、git(Linux 内核开发的版本控制工具)。}
\subsubsection{小组讨论}
\par{在本次开发过程中,小组内逐渐形成了相互学习和讨论的氛围。发现问题时尽早解决
,个人解决不了时召开小组讨论会,共同针对问题进行讨论。在讨论过程中出现分歧时,对
分歧进行分析折中,选择最为适合的解决方案。}

\subsection{企业软件开发方法和规范}
\subsubsection{学会写个人日报和小组周报}
\par{软件工程师日常的工作是设计、编程和测试,但是每日或每段时间进行的工作只是整
个软件产品的一个小的部分。百度规定每个人每天下班前五分钟完成个人日报和每个小组每
周完成小组周报 "帮助 "我们计划当前要作的工作,估算时间,记录项目完成进度,以及计
划下一步的工作和在当天工作中遇到的问题,方便记录和解决问题。}
\subsubsection{学会进行小组计划跟踪}
\par{制定小组计划并对计划进行跟踪,召集小组成员对工作进展情况进行讨论,了解小组
成员间的工作进度,同时对遇到的问题及时解决。}
\subsubsection{学会团队合作}
\par{通过百度“荒岛物品选择”小游戏,充分的体现了团队合作意识以及每个人在团队中的
角色和位置。同时让小组每个成员反思自身在团队中的价值,认识到协作的重要性。也对下
一步的团队合作提出了一定的指导和更高的要求。根据每个人的特长、兴趣爱好调整工作分
配方式,更好的加强团队协作能力、提高团队工作效率。}
\subsubsection{学习软件开发规范}
\par{ 团队在软件开发过程中,根据指导教师的任务安排开展工作,同时学习百度的软件开
发规范和最新的工具、技术,如测试阶段使用GTest等测试工具进行单元测试。}
\subsubsection{获得企业对大学生需求信息}
\par{在实训过程中,指导老师除了对我们进行培训和疑问解答,也向我们介绍IT行业的新
技术、新趋势,同时对我们提出的职业规划和未来的求职问题进行一一解答。让我们了解公
司对大学生的要求和工作后的状况。}

\section{个人收获}
	实训期间,让我学到了很多东西,不仅使我在理论上对IT领域有了全新的认识,在实践
	能力上也得到了提高,真正地做到了学以致用,更学到了很多做人的道理,对我来说受
	益非浅。除此以外,我还学会了如何更好地与别人沟通,如何更好地去陈述自己的观点
	,如何说服别人认同自己的观点。第一次亲身感受到理论与实际的相结合,让我大开眼
	界。也是对以前所学知识的一个初审吧!这次实训对于我以后学习、找工作也真是受益
	菲浅,在短短的二十几天中相信这些宝贵的经验会成为我今后成功的重要的基石。
\subsection{对软件开发有了更深入的了解}
	作为一名大三的学生,经过差不多三年的在校学习,对程序设计有了理性的认识和理解。
	在校期间,一直忙于理论知识的学习,没有机会也没有相应的经验来参与项目的开发。
	所以在实训之前,软件项目开发对我来说是比较抽象的,一个完整的项目要怎么来分工
	以及完成该项目所要的基本步骤也不明确。 而经过这次实训,让我明白一个完整项目的
	开发,必须由团队来分工合作,并在每个阶段中进行必要的总结与论证。一个完整项目
	的开发它所要经历的阶段包括:需求分析、详细设计、代码实现、测试、操作手册。一
	个项目的开发所需要的财力、人力都是大量的,如果没有一个好的远景规划,对以后的
	开发进度会产生很大的影响,严重时导致在预定时间内不能完成该项目或者完成的项目
	跟原先计划所要实现的项目功能不符合。一份好的项目结构、业务功能和详细设计说明
	书对一个项目的开发有明确的指引作用,它可以使开发人员对这个项目所要实现的功能
	在总体上有具体的认识,并能减少在开发过程中出现不必要的脱节。代码的实现是一个
	项目开发成功与否的关键,可以说,前面所做的事情就是为代码的实现做铺垫。
\subsection{感谢老师的培训}
  在短短几天的实训中每位老师都能充分将自己的知识传授给我们,并且有耐心的给我们
	讲解所有问题,帮助每位同学了解企业的事务,发挥了老师的作用与同学打成一片和睦
	相处。感谢老师在这近一个月来给于的关怀及帮助,使我明白了:先进的科学技术和经
	营管理是推动现代化经济和企业高速发展的两个年轮,二者缺一不可。没有先进的管理
	水平,先进的科学技术无法推广,也不能充分发挥它的作用。希望在今后的生活学习与
	工作中充分发挥自己的作用将所学到的知识运用到生活实践中为企业做出自己的贡献。
\subsection{实训意义}
	可以说在我们毕业之前,组织这次实训课,意义重大,在以后的工作、学习中相信大家
	会非常努力,继续充实、完善自己,让自己争当一名出色的程序员。
\begin{description}
	\item[其一]实训是对每个人综合能力的检验。要想做好任何事,除了自己平时要有一定
	的功底外,我们还需要一定的实践动手能力,操作能力。
	\item[其二]此次实训,我深深体会到了积累知识的重要性。俗话说:“要想为事业多添
	一把火,自己就得多添一捆材”。我对此话深有感触。
	\item[再次]“纸上得来终觉浅,绝知此事要躬行!”在短暂的实习过程中,让我深深的感
	觉到自己在实际运用中的专业知识的匮乏,刚开始的一段时间里,对一些工作感到无从
	下手,茫然不知所措,这让我感到非常的难过。在学校总以为自己学的不错,一旦接触
	到实际,才发现自己知道的是多么少,这时才真正领悟到“学无止境”的含义。这也许是
	我一个人的感觉。
\end{description}

\subsection{实训中遇到的问题}
	基础知识掌握不扎实,调试程序的功底仍需加强,于同学交流方面分歧比较大。
\end{document}
%
%/***************  END OF testdesin.tex  **************/
