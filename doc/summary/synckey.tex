%
%
%/********************************************************************
% * filename:           synckeysummary.tex
% * author:             synckey
% * version:            v1.0
% * datetime:           2011-07-05 21:54:29
% * description:        synckey summary
% ********************************************************************/

\documentclass[12pt, a4paper, titlepage]{article}
\usepackage{titletoc}
\usepackage{fontspec}
\usepackage{graphicx}
\usepackage{array}
\usepackage{longtable}
\usepackage{paralist}
\usepackage{float,longtable}
\usepackage{enumerate}
\usepackage{xcolor,listings}
\usepackage{booktabs, multirow}
\usepackage[pagestyles]{titlesec}
\usepackage{xeCJK}


\titlecontents{section}[4em]{\small}{\contentslabel{3.3em}}{}
	{\titlerule*[0.5pc]{$\cdot$}\contentspage}
\titlecontents{subsection}[7em]{\small}{\contentslabel{3.3em}}{}
	{\titlerule*[0.5pc]{$\cdot$}\contentspage}
\titlecontents{subsubsection}[7em]{\small}{\contentslabel{3.3em}}{}
	{\titlerule*[0.5pc]{$\cdot$}\contentspage}
\renewcommand{\today}{\number\year-\number\month-\number\day}
\setmainfont[BoldFont=SimHei]{SimSun}
\setmonofont{SimSun}
\XeTeXlinebreaklocale"zh"

\newpagestyle{mypagestyle}{
	\sethead{\sectiontitle}{}{$\cdot$~\thepage~$\cdot$
	}
	\setheadrule{1pt}
	\setfoot{}{}{\headrule}
}
\newpagestyle{myempty}{
	\sethead{\sectiontitle}{}{}
	\setheadrule{1pt}
	\setfoot{}{}{\headrule}
}
\renewcommand{\contentsname}{目\quad 录}
\renewcommand{\figurename}{图}
\renewcommand{\tablename}{表}
\renewcommand{\refname}{参考文献}
\pagestyle{mypagestyle}
\makeatletter
\let\@afterindentfalse\@afterindenttrue
\@afterindenttrue
\makeatother%

\usepackage{ulem}
\setlength{\parindent}{2em}%中文缩进两个汉字位
\linespread{1.25}
\usepackage[pagebackref,colorlinks, linkcolor = blue , urlcolor = blue]{hyperref}
\newcommand{\tabincell}[2]{\begin{tabular}{@{}#1@{}}#2\end{tabular}}

\title{软件工程实训个人总结}
\author{王健 \quad 200800300234}
\date{2011-07-10}
\begin{document}
\maketitle
\par{Time flies!!!转眼之间项目实训就要结束了,在老师的指导和鼓励下,我们出色完成
了任务。这一个月过得飞快而又漫长,说它漫长,不是因为很累,而是因为我敢觉我成长了很多。}
\par{首先,我学会了团队合作。我们这是一个团队分工的项目,一个人是很难完成的。从
系统的整体设计到项目中的遇到的每一个问题,我们组都经过充分的讨论和交流,最终确定解决方案。
在交流中能感受到每个人对同一个问题的不同观点和看法,学习他人的思想和反思自己的不足。
同时在一个团队中工作也是很快乐的,我们每个人为一个共同的目标而努力,每个人既要专
注自己的领域,又要考虑与其他组员的合作,当把每个人写的模块组装到一起又能成功运行
时,每个组员都很有成就感。}
\par{从这次项目实训中我也锻炼了自己的能力。以前的基本上都是在看书,学习理论知识
,有些知识,以为自己懂了,就真的懂了。但是在实践中遇到了许多意想不到的问题,这些
都是很具体很细节的问题,通过解决这些问题,我更加深刻理解了C语言的指针和内存操作
。通过写socket程序,我熟悉了TCP和UDP的工作原理,为以后写出更好的C/S程序打下基础
。我们这个程序是用多线程写的,在代码实现的过程中,遇到了很多同步的问题,解决这些
问题的过程也是一个学习的过程。总之,这次实训使我认识到,虽然理论知识很重要,但是
动手的实践也能学到很多书上学不到的知识。理论和实践这两者缺一不可。}
\par{在这一次实训中,我也感受到了百度公司良好的企业文化。百度来的几位老师都非常好。他们不仅
仅教我们技术方面的问题,也教会我们一些团队合作和职业生涯规划的问题。他们每天都会
给我们安排吃水果的时间,虽然是一个很小的举动,但是从一个侧面反映了百度公司的文化
和对人才的态度。与其他IT公司不同,百度非常重视人才的生活和工作的平衡,员工既要有
生活,又要有工作,这是很难得的。}
\par{}

\end{document}
%
%/***************  END OF testdesin.tex  **************/
